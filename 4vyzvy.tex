\chapter{Shrnutí}

Nanotechnologie jsou součástí informační  revoluce. Jejich možnosti jsou obrovské. Konvenční materiály přivedly lidstvo do 20. století. Století páry překonala elektrifikace a na tu navazuje digitalizace. Energetika se odklání od fosilních paliv a udržitelný rozvoj se přiklání k obnovitelným zdrojům, které díky nanotechnologiím zvyšují svoji účinnost - fotovoltaické elektrárny, možnosti uskladnění energie v bateriích \cite{he3da_baterie}, snížení ztrát při transportu elektrické energie. Stavitelství a materiálový průmysl produkuje materiály s obdivuhodnými vlastnostmi. Konstrukce, které jsou pevné a zároveň lehké. Letadla, která díky nízké hmotnosti, snižují spotřebu paliv. Antikorozní povrchy prodlužují životnost mnoha stavebních materiálů. Funkční materiály v textilním průmyslu zvyšují odolnost vůči zápachu či skvrnám. Elektroniku si dnes vezmeme do kapsy a můžeme vyrazit na výlet. Působení léčiv můžeme prodloužit a snížit tak akutní dávku nebo je přímo zacílit do nemocné tkáně a ochránit tak zdravé buňky. Nové obaly s nanotechnologiemi ochrání potraviny před mikroby a zabrání tak rychlé skáze. Nanotechnologie jsou všude kolem nás, ať si to uvědomujeme či nikoliv. Konvenční materiály přivedly lidstvo na okraj možného, přes který s nanotechnologiemi hledíme vstříc novým výzvám.\\

Velké investice i soukromého sektoru do výzkumu nanotechnologií svědčí o předpokladu vysoké návratnosti těchto investic. Evropská unie, Amerika i národní strategie podporují rozvoj nanotechnologií a současně vzniká nová legislativa a studie toxicity a ekotoxicity těchto materiálů budoucnosti.\\

Rizika nanotechnologií jsou asi nejrychleji sledovanými riziky průmyslových revolucí v historii. Současná globalizace umožňuje rychlé sdílení informací o aplikaci nanomateriálů, ale i o rizikách jejich využívání. Je sledována ekotoxicita různých expozic a dávek nanomateriálů. Velkým otazníkem jsou dlouhodobé studie toxicity a ekotoxicity. Tyto studie nelze v krátkém časovém úseku zrealizovat a vyhodnotit. Velkým pomocníkem jsou proto matematické modely, které se snaží predikovat různé chování látek v biosféře. Systematizace predikce rizik nanotechnologií je významným prvkem hodnocení vlivu nanotechnologií, protože konvenční metody hodnocení rizik nedokáží postihnout specifiku nanomateriálů v porovnání s klasickými materiály. A proto důkladná studie nanomateriálů a hodnocení jejich vlivu na biosféru je nejdůležitějším aspektem přístupu k nobým budoucím technologiím.\\



